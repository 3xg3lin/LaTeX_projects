
\documentclass[12pt,letterpaper]{article} % declare the document type and others.
\usepackage{graphicx}  % for loading externel package

\graphicspath{{images/},{pictures/}}  % graphicx package path declaration

\title{My first LaTeX document} % maketitle command content start here,
\author{Kerem Utku Gül\thanks{thanks}} % and continous here
\date{November 2025} % and here.

% until here we say preamble for document
\begin{document}  % document starting here.
\maketitle % if you want to use title, author and date in your document.
First document. This is simple example, with no extra parameters or packages included.
% This line here is a comment. It will not be typeset in a document.

Some of the \textbf{greatest}  % bold 
discoveries in \underline{science}  % underline
were made by \textbf{\textit{accident}}.  % italic

Some of the \emph{greatest}  % emphasize is similar to italics
discoveries in \underline{\emph{science}}
were made by \textbf{\textit{\emph{accident}}}. % but this behave different

\begin{figure}[h]  % \begin{environment-name} is the core consept so in this sample I used figure.
    \centering
    \includegraphics[width=0.75\textwidth]{f1}
    \caption{A nice plot}
    \label{figure:mesh1}  % When I wanted to make a reference in the document, I used the \label command.
\end{figure}

As you can see figure \ref{figure:mesh1}, the function grows near the origin.  %\ref command replaced with the number corresonding in the label.
this is on page \pageref{figure:mesh1}.

\begin{itemize}  % making unoredered list
    \item The individual entries are dedicated with a black dot,a so-called bullet.
    \item The text in the entries may be of any lenght.
\end{itemize}

\begin{enumerate} % making ordered list
    \item This is the first entry in our list.
    \item The list numbers increase with each entry we added.
\end{enumerate}

In pyshics, the mass-energy equivalence is stated  
by the equation $E=mc^2$ , discovered in 1905 by % inline math mode in used here
Albert Einstein. % for using inline math mode we can use;\( ... \), $ ... $ or \begin{math} ... \end{math} pairs.

\begin{math}
E=mc^2
\end{math} is typeset in a paragraph using 
inline math mode---as is $E=mc^2$, and so too is \(E=mc^2\).

The mass-energy equivalence is described by the famous equation
\[ E=mc^2 \] discovered in 1905 by 
Albert Einstein. %  for using display math mode we can use; \[ ... \], \begin{displaymath} ... \end{displaymath} or \begin{equation} ... \end{equation} pairs.

In natural units ($c = 1$), the formula expresses the identity
\begin{equation}
E=m
\end{equation}

Subscripts in math mode are written as $a_b$ and superscripts are written as $a^b$. These can be combined and nested to write expressions such as

\[ T^{i_1 i_2 \dots i_p}_{j_1 j_2 \dots j_q} = T(x^{i_1},\dots,x^{i_p},e_{j_1},\dots,e_{j_q}) \]

We write integrals using $\int$ and fractions using $\frac{a}{b}$. Limits are placed on integrals using superscripts and subscripts:

\[ \int_0^1 \frac{dx}{e^x} =  \frac{e-1}{e} \]

Lower case Greek letters are written as $\omega$ $\delta$ etc. while upper case Greek letters are written as $\Omega$ $\Delta$.

Mathematical operators are prefixed with a backslash as $\sin(\beta)$, $\cos(\alpha)$, $\log(x)$ etc.
\end{document}
